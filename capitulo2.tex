\chapter{Definição do Problema de Estudo}
Neste capítulo você deve apresentar uma descrição do problema de estudo.
Deve conter uma explicação clara do problema e porque ele ainda é atual e necessita de solução.
Ao fim deste capítulo você define o seu objetivo geral e os objetivos específicos. 
%TODO: Falar sobre o sistema de tração do veículo elétrico. Também falar sobre esse veículo no geral.
%TODO: Inserir referência.
A falta de infraestrutura para o carregamento de veículos elétricos é citada em {Inserir ref} como
um dos principais gargalos para a adoção em massa desse tipo de véiculo. Considerando esse desafio,
o seguinte trabalho propõe o projeto de um carregador que seja adequado á realidade do sistema de distribuição
brasileiro. Dessa forma, escolhe-se trabalhar com um sistema de carregamento trifásico.
%TODO: Comentar sobre carregadores onboard no geral.


\section{Objetivo Geral}
O objetivo geral do trabalho é projetar o circuito de um carregador de veículo elétrico \textit{onboard} trifásico, composto por um retificador PFC e um conversor CC-CC isolado.
\section{Objetivos Específicos}
\begin{itemize}
    \item Realizar revisão bibliográfica sobre as diferentes topologias utilizadas no carregamento de carros elétricos;
    \item Projeto do circuito retificador PFC trifásico;
    \item Sintonia do controlador do retificador PFC trifásico;
    \item Projeto do conversor CC-CC \textit{Phase-Shifted Full Bridge};
    \item Projeto do conversor CC-CC \textit{Dual Active Bridge}(DAB);
    \item Comparação entre o \textit{Phase-Shifted Full Bridge} e o DAB.
\end{itemize}