\chapter{Revisão Bibliográfica}
Neste capítulo reúne-se as fontes de pesquisa que vão fornecer o embasamento teórico para o seu 
trabalho. É na revisão da literatura onde deve-se apresentar os trabalhos relacionados ao tema 
ou cuja solução pode ser aplicada ao problema de estudo. A revisão bibliográfica não é um 
conjunto de trecho de obras, mas uma apresentação pessoal e crítica dos trabalhos anteriormente 
realizados. Recomenda-se realizar uma revisão bibliográfica com os seguintes passos:
\begin{enumerate}
	\item Escolha boas referências – Encontre autores e obras especializadas no tema do seu trabalho de TCC.
	\item Analise as referências – Verifique se as referências têm relação direta com o recorte do seu problema.
	\item Crie um padrão – Esse padrão pode ser cronológico, ou pode ser classificada por tipo de solução. Ex. Um problema de controle pode ser abordado por técnicas de controle lineares e não-lineares. Assim, a sua revisão pode ser dividida por ordem cronológica em que os trabalhos foram apresentados ou classificados por soluções de controle lineares e não-lineares. 
	\item Faça uma análise crítica das referências – Leia as obras, analise-as e faça anotações. Sintetize o conteúdo de cada obra e reflita sobre cada teoria/solução apresentada de modo a destacar a sua visão crítica. Dessa forma, a sua revisão não será um mero enxerto de passagens da literatura.
\end{enumerate}

Em relação ao retificador PFC trifásico, tanto \cite{3phPlecs} quanto \cite{WANG2013/03} 
realizam o controle do conversor CA-CC por meio da transformada de Park. Nessa metodologia,
um \textit{Phase Locked Loop} (PLL) captura a referência de fase da tensão de entrada, que em seguida
 é utilizada na transformação para o sistema de coordenadas dq0.Ademais, duas malhas de controle, uma externa, referente à tensão de saída do retificador e uma interna,
 que controla as componentes de eixo direto e em quadratura da corrente de entrada. Como o objetivo é
 controlar o fator de potência, a componente de corrente em quadratura é ajustada para ser nula,

 Conforme a figura \ref{fig:controlepfc3ph}, o controle do PFC trifásico 

 \begin{figure}
 	\centering
	\caption{Esquema de controle do PFC trifásico.}
	\includegraphics[width=0.8\textwidth]{./Figuras/controlepfc3ph.png}
	\legend{Fonte: \cite{3phPlecs}.}
	\label{fig:controlepfc3ph}
 \end{figure}