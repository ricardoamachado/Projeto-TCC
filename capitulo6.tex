\chapter{Cronograma e Recursos}
\section{Cronograma}

Cada etapa do projeto deve produzir algum resultado. Assim, o processo de planejamento deve
produzir um cronograma de metas, indicando quando cada objetivo deve ser concluído. Tal
cronograma deve conter objetivos mensuráveis, em que cada etapa produz um resultado específico.
Por exemplo, se você pretende utilizar três semanas em leitura de obras, o resultado deve ser
uma seção de revisão bibliográfica para o seu trabalho de final de graduação e não apenas a
leitura das obras. Esta etapa deve conter uma tabela de tarefas a serem realizadas dentro de um
período de tempo determinado para alcançar cada objetivo específico.

\section{Recursos}

Apresente uma estimativa do que será necessário para realizar o projeto. Indique os recursos de
hardware, software ou orçamento necessário para realizar o projeto. Esta etapa é importante
para destacar a viabilidade do projeto. Dessa forma, qualquer ferramenta necessária para
execução do projeto e sua disponibilidade deve estar incluída nesta seção.

\begin{table}[h]
    \centering
    \caption{Lista de recursos para o trabalho de conclusão de curso.}
    \label{tab:recursos}
    \begin{tabular}{|l|l|}
        \hline
        Recurso          & Categoria      \\ \hline
        Computador       & Recurso Físico \\ \hline
        Osciloscópio     & Recurso Físico \\ \hline
        Fonte CC         & Recurso Físico \\ \hline
        Carga Eletrônica & Recurso Físico \\ \hline
        MATLAB           & Software       \\ \hline
        PSIM             & Software       \\ \hline
    \end{tabular}
\end{table}