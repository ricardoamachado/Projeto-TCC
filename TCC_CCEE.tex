\documentclass{CCEE_UFBA_Class}

%%%%%%%%%%%%%%%%%%%%%%%%%%%%%%%%%%%%%%%%
%	Arquivo criado pelo CCEE utilizando
%   como base a classe AbnTeX2.
%   Autores: André Tahim e Cristiane Paim
%	Universidade Federal da Bahia
%	Escola Politécnica
%   Deptº de Eng. Elétrica e de Computação
%	07/01/2021
%%%%%%%%%%%%%%%%%%%%%%%%%%%%%%%%%%%%%%%%
% ---
% Informações de dados para CAPA e FOLHA DE ROSTO
% -------------------------------------------------------------------------------
\titulo{Título do Projeto de TCC}
\autor{Ricardo Augusto de Araújo Machado}
\orientador{Orientador: Prof. Dr. André Pires Nóbrega Tahim}
%\coorientador{Coorientador: Prof. Dr. Y} % Comente se nao existir.
\local{Salvador-Ba -- Brasil}
\data{2025}
\tipotrabalho{Relatório técnico}
% -------------------------------------------------------------------------------


\begin{document}
	
% Seleciona o idioma do documento (conforme pacotes do babel)
%\selectlanguage{english}
\selectlanguage{brazil}

% Retira espaço extra obsoleto entre as frases.
\frenchspacing 

% ----------------------------------------------------------
% ELEMENTOS PRÉ-TEXTUAIS
% ----------------------------------------------------------
% \pretextual

% ---
% Capa
% ---
\imprimircapa

% ---
% Folha de rosto
% (o * indica que haverá a ficha bibliográfica)
% ---
%\imprimirfolhaderosto*
% ---
\cleardoublepage
% ---
% inserir o sumario
% ---
\pdfbookmark[0]{\contentsname}{toc}
\tableofcontents*
\cleardoublepage
% ---


% ----------------------------------------------------------
% ELEMENTOS TEXTUAIS
% ----------------------------------------------------------
\textual
	
%%%%%%%%%%%%%%
%	CAPÍTULOS
%%%%%%%%%%%%%%
\chapter{Introdução}
Atualmente, o setor de transportes corresponde a 30\% e 31\% da matriz energética dos Estados Unidos e dos países da União Europeia, respectivamente \cite{eia_transportation_energy} e \cite{eurostat_transport_energy}. Tendo em vista que é necessário reduzir a emissão de gases do efeito estufa significativamente nos próximos anos, a eletrificação do setor automobilístico é uma medida importante para atingir o objetivo supracitado. Ao mesmo tempo, a ausência de uma infraestrutura de carregamento robusta e a falta de autonomia são os maiores gargalos para a adoção em massa de veículos elétricos.


O desenvolvimento de uma infraestrutura de carregamento deve ser planejado para não sobrecarregar a rede elétrica e respeitar tanto a norma da Aneel Nº 414/201, que limita o fator de potência para valores a partir de 0,92 quanto as normas relativas à distorção harmônica máxima permitida como, por exemplo, a norma europeia IEC 61000-3-2 e a norma chinesa GB 17625.1-2022 . Dessa forma, um carregador de veículo elétrico obrigatoriamente precisa apresentar um estágio ativo de correção do fator de potência. Ademais, a fim de garantir uma maior autonomia para essa categoria de veículos, é necessário que os carregadores forneçam uma potência suficientemente alta.

Os carregadores de veículos elétricos podem ser \emph{on-board} ou \emph{off-board}. No primeiro caso, o veículo é conectado diretamente à rede elétrica e o circuito de potência interno do automóvel realiza a retificação da tensão de entrada CA para CC. No último caso, as estações de carregamento realizam o processo de retificação e conversão de energia e fornecem para o veículo uma tensão de alimentação CC.

Considerando os desafios existentes para a utilização em massa de veículos elétricos, este trabalho visa projetar e simular um carregador \emph{on-board} trifásico bidirecional completo, incluindo tanto o estágio do PFC quanto o estágio do conversor CC-CC.

\chapter{Definição do Problema de Estudo}
Neste capítulo você deve apresentar uma descrição do problema de estudo.
Deve conter uma explicação clara do problema e porque ele ainda é atual e necessita de solução.
Ao fim deste capítulo você define o seu objetivo geral e os objetivos específicos. 
%TODO: Falar sobre o sistema de tração do veículo elétrico. Também falar sobre esse veículo no geral.
%TODO: Inserir referência.
A falta de infraestrutura para o carregamento de veículos elétricos é citada em {Inserir ref} como
um dos principais gargalos para a adoção em massa desse tipo de véiculo. Considerando esse desafio,
o seguinte trabalho propõe o projeto de um carregador que seja adequado á realidade do sistema de distribuição
brasileiro. Dessa forma, escolhe-se trabalhar com um sistema de carregamento trifásico.
%TODO: Comentar sobre carregadores onboard no geral.


\section{Objetivo Geral}
O objetivo geral do trabalho é projetar o circuito de um carregador de veículo elétrico \textit{onboard} trifásico, composto por um retificador PFC e um conversor CC-CC isolado.
\section{Objetivos Específicos}
\begin{itemize}
    \item Realizar revisão bibliográfica sobre as diferentes topologias utilizadas no carregamento de carros elétricos;
    \item Projeto do circuito retificador PFC trifásico;
    \item Sintonia do controlador do retificador PFC trifásico;
    \item Projeto do conversor CC-CC \textit{Phase-Shifted Full Bridge};
    \item Projeto do conversor CC-CC \textit{Dual Active Bridge}(DAB);
    \item Comparação entre o \textit{Phase-Shifted Full Bridge} e o DAB.
\end{itemize}
\chapter{Revisão Bibliográfica}
%% Visão geral sobre carregadores de veículos elétricos.

O artigo de \cite{Yuan:2021} faz uma revisão sistemática sobre carregadores \textit{onboard}
bidirecionais e aborda sobre os principais modelos comerciais presentes no mercado. Dentre os
sistemas de carregamento trifásico citados há o carregador trifásico da \textit{Current Ways}
que oferece uma potência de saída de até 26,4 kW, uma tensão de saída na faixa de 250 V a 425 V
e eficiência em torno de 96 \%. A estrutura interna do carregador é composta por retificador
PFC ativo de ponte completa e um conversor CC-CC isolado DAB. Já o carregador da \textit{EATON}
fornece uma potência de saída de até 22 kW com uma tensão de saída na faixa de 225 V a 500 V e
eficiência de 95 \%. A empresa não detalha sobre a composição interna do carregador.

O texto de \cite{Yuan:2021} também revisa as principais tecnologias que são utilizadas
atualmente nos carregadores.

\cite{Baharom:2024} comentam sobre futuros avanços tecnológicos no carregamento de veículos elétricos. O texto aborda especificamente sobre os desafios de desenvolver um carregador onboard bidirecional, que incluem a dificuldade de gerenciar o fluxo de potência em ambos os sentidos, a complexidade necessária para o sistema de controle e a integração desses sistemas com a rede elétrica existente. Ademais, o artigo aborda sobre o carregamento indutivo sem fio e justifica a necessidade em pesquisa nessa tecnologia com base na maior segurança e menor necessidade de manutenção obtida. Ao mesmo tempo, os autores reconhecem que a limitação na transferência de potência sem fios e a baixa eficiência inviabilizam esse sistema na prática.

De acordo com \cite{Kumar:2021}, os convesores CA-CC bidirecionais, que estão presentes em
carregadores de carros elétricos, apresentam desafios em relação a sincronização de frequência
e fase, controle do fator de potência e qualidade de isolação.

%% Visão sobre o PFC.

Os \textit{application notes} \cite{onsemi_hbd853} e \cite{ti_zhcp224} apresentam uma visão
geral sobre o controle do fator de potência ativo com um conversor boost monofásico. Esse
conversor pode operar tanto em modo de condução contínua (CCM) quanto em modo de condução
crítica (CRM). Em CCM, o conversor é controlado através do controle pelo modo de corrente
média. Nesse cenário, a malha de controle externa é lenta e tem o papel de ajustar o valor de
tensão de saída enquanto que a malha de controle de corrente(interna) é mais rápida e visa
fazer com que a corrente média do indutor siga a forma de onda de uma entrada de referência
senoidal que esteja em fase com a tensão da rede. Já a operação em CRM é adequada apenas para
uma potência de saída abaixo de 300 W e caracteriza-se pelo controle através de um sinal PWM
com frequência de cheaveamento variável e tempo ON constante.

O artigo da ONSEMI \cite{onsemi_h2ptoday2102} aborda as principais topologias de retificadores
PFC trifásicos. A primeira topologia citada é o retificador PWM Vienna (Fig.
\ref{fig:pfc_vienna}), que é caracterizada como uma ponte retificadora trifásica conectada a
uma conversor boost por fase. A topologia Vienna tem como principal vantagem o uso de apenas um
transistor por fase, o que simplica significativamente o controle do conversor e o cheaveamento
em três níveis, ao mesmo tempo, o alto número de diodos nessa topologia implica em maiores
perdas. Já topologia T-NPC (Fig. \ref{fig:pfc_tnpc}) é derivada da Vienna e consiste
basicamente numa ponte de diodos trifásica conectada a seis transistores. A principal vantagem
dessa topologia é o menor número de componentes em relação à anterior. Ambas as topologias
citadas são unidirecionais. Ademais, a topologia T-NPC pode ser implementada apenas com
transistores, o que a torna bidirecional.

\begin{figure}[h]
	\centering
	\begin{minipage}{0.45\textwidth}
		\centering
		\caption{Retificador PFC Vienna.}
		\includegraphics[width=\textwidth]{./Figuras/PFC_Vienna.png}
		\legend{Fonte: \cite{onsemi_h2ptoday2102}}
		\label{fig:pfc_vienna}
	\end{minipage}
	\hfill
	\begin{minipage}{0.45\textwidth}
		\centering
		\caption{Retificador T-NPC.}
		\includegraphics[width=\textwidth]{./Figuras/PFC_TNPC.png}
		\legend{Fonte: \cite{onsemi_h2ptoday2102}}
		\label{fig:pfc_tnpc}
	\end{minipage}
\end{figure}

A principal topologia de retificador PFC trifásico é o \textit{Six-switch rectifier} vista na
Figura \ref{fig:pfc_six_switch}, que consiste numa ponte trifásica formada por seis
transistores de tensão nominal na faixa de 900 V a 1200 V. A principal vantagem dessa topologia
é a bidirecionalidade, inclusive porque o inversor trifásico a seis transistores é
frequentemente utilizado no acionamento de motores elétricos \cite{onsemi_h2ptoday2102}. Ao
mesmo tempo, esse retificador apresenta maior interferência eletromagnética que as outras
opções citadas e opera com chaveamento em dois níveis.

\begin{figure}
	\centering
	\caption{Retificador trifásico PFC de ponte completa.}
	\includegraphics[width=0.7\textwidth]{./Figuras/retificador_six_switch.png}
	\legend{Fonte: \cite{WANG2013/03}}
	\label{fig:pfc_six_switch}
\end{figure}

%% Controle do PFC Trifásico

Em relação ao retificador PFC trifásico, tanto \cite{3phPlecs} quanto \cite{WANG2013/03}
realizam o controle do conversor CA-CC por meio da transformada de Park. Nessa metodologia, um
\textit{Phase Locked Loop} (PLL) captura a referência de fase da tensão de entrada, que em
seguida é utilizada na transformação para o sistema de coordenadas dq0.Ademais, duas malhas de
controle, uma externa, referente à tensão de saída do retificador e uma interna, que controla
as componentes de eixo direto e em quadratura da corrente de entrada. Como o objetivo é
controlar o fator de potência, a componente de corrente em quadratura é ajustada para ser nula.

A nota de aula \cite{dq0_transform} explica sobre como aplicar a Transformada de Park em um
modelo de circuito trifásico modelado em espaços de estados. Já o livro \cite{} aborda sobre os
quadros de referência \(\alpha \beta\) e dq0, utilizados na Transformada de Clarke e na
Transformada de Park, respectivamente.

%% Conversor CC-CC isolado

O \textit{application note} da \cite{Infineon_CoolMOS_CFD7A} aborda sobre as principais
topologias de conversores CC-CC isolados utilizados nos carregadores \textit{onboard} de
veículos elétricos. O \textit{Phase Shifted Full Bridge} é composto por uma ponte completa
ativa, um transformador e uma ponte retificadora e caracteriza-se pela alta eficiência, uso de
comutação suave e controle do fluxo de potência pelo ajuste da desafasagem de tensão entre os
dois braços da ponte de transistores. O \textit{Dual Active Bridge} substitui os diodos da
ponte retificadora por transistores, permitindo a bidirecionalidade. Para garantir maior
eficiência, o conversor ressonante LLC é utilizado por conta da capacidade de operar com
\textit{Zero Voltage Switching}(ZVS). Ao mesmo tempo, os conversores ressonantes apresentam
maior complexidade, pois são controlados com frequência variável ao invés de uma razão cíclica
variável.


\chapter{Solução e Avaliação da Proposta}
\section{Solução Proposta}
Esta seção é destinada a explicação de como você resolverá o problema, ou como você pretende realizar o projeto. Neste estágio é preciso estar claro o que você irá fazer, mas flexível o suficiente para adaptar a solução a adversidades encontradas durante a realização do projeto. A escrita da proposta da solução não o previne do insucesso, mas permite que você identifique diversos problemas com antecedência.
Note que a metodologia descrita aqui é dependente do tipo de projeto do seu TCC. Você deve discutir com seu orientador sobre isso. 

\section{Avaliação da Solução}
Esta seção deve conter uma explicação de como será feita a avaliação da solução uma vez que ela estiver concluída. O método de avaliação é dependente do projeto e o seu orientador pode guiá-lo para a forma de avaliação mais adequada.

\chapter{Resultados Preliminares}
Este capítulo é facultativo! Caso o seu trabalho tenha alcançado algum resultado preliminar, ele deve ser incluído aqui. Deve-se destacar que para qualquer simulação (ou experimento) o aparato utilizado e os procedimentos seguidos para obtenção dos resultados devem ser detalhadamente apresentados. \emph{A descrição da sua simulação (ou experimento) deve conter informações suficientes para que o leitor possa repliclá-lo}. Frequentemente figuras bem elaborados conseguem condensar e dar uma ideia esclarecedora de como os resultados foram obtidos, tais como: fotos de bancada, diagrama de blocos, esquemáticos, fluxogramas, etc. É importante para qualquer engenheiro/pesquisador adquirir a habilidae de descrever/ilustrar a essência de como os resultados foram obtidos.

\chapter{Cronograma e Recursos}
 \section{Cronograma}

Cada etapa do projeto deve produzir algum resultado. Assim, o processo de planejamento deve produzir um cronograma de metas, indicando quando cada objetivo deve ser concluído. Tal cronograma deve conter objetivos mensuráveis, em que cada etapa produz um resultado específico. Por exemplo, se você pretende utilizar três semanas em leitura de obras, o resultado deve ser uma seção de revisão bibliográfica para o seu trabalho de final de graduação e não apenas a leitura das obras.  
Esta etapa deve conter uma tabela de tarefas a serem realizadas dentro de um período de tempo determinado para alcançar cada objetivo específico.

\section{Recursos} 

Apresente uma estimativa do que será necessário para realizar o projeto. Indique os recursos de hardware, software ou orçamento necessário para realizar o projeto. Esta etapa é importante para destacar a viabilidade do projeto. Dessa forma, qualquer ferramenta necessária para execução do projeto e sua disponibilidade deve estar incluída nesta seção. 


% ----------------------------------------------------------
% ELEMENTOS PÓS-TEXTUAIS
% ----------------------------------------------------------
\postextual

\bibliographystyle{./Auxiliares/abntex2-alf}
\bibliography{./Auxiliares/ref-TCC}

%% ---
% Inicia os apêndices
% ---
\begin{apendicesenv}

% Imprime uma página indicando o início dos apêndices
%\partapendices

% ----------------------------------------------------------
\chapter{Informações e Dicas \LaTeX}
% ----------------------------------------------------------

Este apêndice contém algumas dicas para escrita de um documento técnico em \LaTeX no formato ABNT.

\section{Inserindo Citações}

\textbf{Latex:}
\begin{verbatim}
Imagine que isso é um texto que necessita de uma citação
\cite{Kasper:2014}. Talvez mais uma informação seja necessária
\cite{Liu:2011}.
\end{verbatim}

\textbf{Resultado:}\\
Imagine que isso é um texto que necessita de uma citação \cite{Kasper:2014}. Talvez mais uma informação seja necessária \cite{Liu:2011}.

\section{Inserindo Equações}

\textbf{Latex:}
\begin{verbatim}
\begin{equation}
\cos (2\theta) = \cos^2 \theta - \sin^2 \theta
\end{equation}.
\end{verbatim}

\textbf{Resultado:}\\
\begin{equation}
\cos (2\theta) = \cos^2 \theta - \sin^2 \theta
\end{equation}.
\newpage
\section{Citando Equações}

Para citar uma equação é necessário dar um nome pra equação (um label), no exemplo abaixo a equação foi rotulada de \textit{eq:cos2theta}:

\textbf{Latex:}
\begin{verbatim}
\begin{equation}
\label{eq:cos2theta}
\cos (2\theta) = \cos^2 \theta - \sin^2 \theta
\end{equation}.

Para citá-la apenas utilize o comando eqref. 

Ex. Substituindo \eqref{eq:cos2theta} em...  
\end{verbatim}



\textbf{Resultado:}\\
\begin{equation}
\label{eq:cos2theta}
\cos (2\theta) = \cos^2 \theta - \sin^2 \theta
\end{equation}.

Para citá-la apenas utilize o comando eqref. 

Ex. Substituindo \eqref{eq:cos2theta} em... 

\newpage
\section{Inserindo Figuras}

\subsection{Inserindo uma Figura}
\textbf{Latex:}
\begin{verbatim}
\begin{figure}[htb]
\caption{\label{fig:farol-da-barra}Farol da Barra}
\begin{center}
\includegraphics[width=0.7\linewidth]{./Figuras/farol-da-barra.jpg}
\end{center}
\legend{Fonte: os autores}
\end{figure}
\end{verbatim}

\textbf{Resultado:}
\begin{figure}[htb]
	\caption{\label{fig:farol-da-barra}Farol da Barra}
	\begin{center}
		\includegraphics[width=0.7\linewidth]{./Figuras/farol-da-barra.jpg}
	\end{center}
	\legend{Fonte: os autores}
\end{figure}

Eu agora irei citar a Figura \ref{fig:farol-da-barra}

\newpage
\subsection{Inserindo Figuras Lado a Lado}

\textbf{Latex:}
\begin{verbatim}
\begin{figure}[!htb]
\label{teste}
\centering
\begin{minipage}{0.45\textwidth}
\centering
\caption{Imagem 1 da minipage} \label{fig:poli1}
\includegraphics[width=1\linewidth]{./Figuras/poli1}
\legend{Fonte: Produzido pelos autores}
\end{minipage}
\hfill
\begin{minipage}{0.45\textwidth}
\centering
\caption{Imagem 2 da minipage} \label{fig:poli2}
\includegraphics[width=1\linewidth]{./Figuras/poli1}
\legend{Fonte: o autor}
\end{minipage}
\end{figure}
\end{verbatim}


\textbf{Resultado:}
\begin{figure}[!htb]
	\label{teste}
	\centering
	\begin{minipage}{0.45\textwidth}
		\centering
		\caption{Imagem 1 da minipage} \label{fig:poli1}
		\includegraphics[width=1\linewidth]{./Figuras/poli1}
		\legend{Fonte: Produzido pelos autores}
	\end{minipage}
	\hfill
	\begin{minipage}{0.45\textwidth}
		\centering
		\caption{Imagem 2 da minipage} \label{fig:poli2}
		\includegraphics[width=1\linewidth]{./Figuras/poli1}
		\legend{Fonte: o autor}
	\end{minipage}
\end{figure}

\end{apendicesenv}

\end{document}