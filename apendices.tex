% ---
% Inicia os apêndices
% ---
\begin{apendicesenv}

% Imprime uma página indicando o início dos apêndices
%\partapendices

% ----------------------------------------------------------
\chapter{Informações e Dicas \LaTeX}
% ----------------------------------------------------------

Este apêndice contém algumas dicas para escrita de um documento técnico em \LaTeX no formato ABNT.

\section{Inserindo Citações}

\textbf{Latex:}
\begin{verbatim}
Imagine que isso é um texto que necessita de uma citação
\cite{Kasper:2014}. Talvez mais uma informação seja necessária
\cite{Liu:2011}.
\end{verbatim}

\textbf{Resultado:}\\
Imagine que isso é um texto que necessita de uma citação \cite{Kasper:2014}. Talvez mais uma informação seja necessária \cite{Liu:2011}.

\section{Inserindo Equações}

\textbf{Latex:}
\begin{verbatim}
\begin{equation}
\cos (2\theta) = \cos^2 \theta - \sin^2 \theta
\end{equation}.
\end{verbatim}

\textbf{Resultado:}\\
\begin{equation}
\cos (2\theta) = \cos^2 \theta - \sin^2 \theta
\end{equation}.
\newpage
\section{Citando Equações}

Para citar uma equação é necessário dar um nome pra equação (um label), no exemplo abaixo a equação foi rotulada de \textit{eq:cos2theta}:

\textbf{Latex:}
\begin{verbatim}
\begin{equation}
\label{eq:cos2theta}
\cos (2\theta) = \cos^2 \theta - \sin^2 \theta
\end{equation}.

Para citá-la apenas utilize o comando eqref. 

Ex. Substituindo \eqref{eq:cos2theta} em...  
\end{verbatim}



\textbf{Resultado:}\\
\begin{equation}
\label{eq:cos2theta}
\cos (2\theta) = \cos^2 \theta - \sin^2 \theta
\end{equation}.

Para citá-la apenas utilize o comando eqref. 

Ex. Substituindo \eqref{eq:cos2theta} em... 

\newpage
\section{Inserindo Figuras}

\subsection{Inserindo uma Figura}
\textbf{Latex:}
\begin{verbatim}
\begin{figure}[htb]
\caption{\label{fig:farol-da-barra}Farol da Barra}
\begin{center}
\includegraphics[width=0.7\linewidth]{./Figuras/farol-da-barra.jpg}
\end{center}
\legend{Fonte: os autores}
\end{figure}
\end{verbatim}

\textbf{Resultado:}
\begin{figure}[htb]
	\caption{\label{fig:farol-da-barra}Farol da Barra}
	\begin{center}
		\includegraphics[width=0.7\linewidth]{./Figuras/farol-da-barra.jpg}
	\end{center}
	\legend{Fonte: os autores}
\end{figure}

Eu agora irei citar a Figura \ref{fig:farol-da-barra}

\newpage
\subsection{Inserindo Figuras Lado a Lado}

\textbf{Latex:}
\begin{verbatim}
\begin{figure}[!htb]
\label{teste}
\centering
\begin{minipage}{0.45\textwidth}
\centering
\caption{Imagem 1 da minipage} \label{fig:poli1}
\includegraphics[width=1\linewidth]{./Figuras/poli1}
\legend{Fonte: Produzido pelos autores}
\end{minipage}
\hfill
\begin{minipage}{0.45\textwidth}
\centering
\caption{Imagem 2 da minipage} \label{fig:poli2}
\includegraphics[width=1\linewidth]{./Figuras/poli1}
\legend{Fonte: o autor}
\end{minipage}
\end{figure}
\end{verbatim}


\textbf{Resultado:}
\begin{figure}[!htb]
	\label{teste}
	\centering
	\begin{minipage}{0.45\textwidth}
		\centering
		\caption{Imagem 1 da minipage} \label{fig:poli1}
		\includegraphics[width=1\linewidth]{./Figuras/poli1}
		\legend{Fonte: Produzido pelos autores}
	\end{minipage}
	\hfill
	\begin{minipage}{0.45\textwidth}
		\centering
		\caption{Imagem 2 da minipage} \label{fig:poli2}
		\includegraphics[width=1\linewidth]{./Figuras/poli1}
		\legend{Fonte: o autor}
	\end{minipage}
\end{figure}

\end{apendicesenv}