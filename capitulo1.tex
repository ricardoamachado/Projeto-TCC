\chapter{Introdução}
Atualmente, o setor de transportes corresponde a 30\% e 31\% da matriz energética dos Estados Unidos e dos países da União Europeia, respectivamente \cite{eia_transportation_energy} e \cite{eurostat_transport_energy}. Tendo em vista que é necessário reduzir a emissão de gases do efeito estufa significativamente nos próximos anos, a eletrificação do setor automobilístico é uma medida importante para atingir o objetivo supracitado. Ao mesmo tempo, a ausência de uma infraestrutura de carregamento robusta e a falta de autonomia são os maiores gargalos para a adoção em massa de veículos elétricos.


O desenvolvimento de uma infraestrutura de carregamento deve ser planejado para não sobrecarregar a rede elétrica e respeitar tanto a norma da Aneel Nº 414/201, que limita o fator de potência para valores a partir de 0,92 quanto as normas relativas à distorção harmônica máxima permitida como, por exemplo, a norma europeia IEC 61000-3-2 e a norma chinesa GB 17625.1-2022 . Dessa forma, um carregador de veículo elétrico obrigatoriamente precisa apresentar um estágio ativo de correção do fator de potência. Ademais, a fim de garantir uma maior autonomia para essa categoria de veículos, é necessário que os carregadores forneçam uma potência suficientemente alta.

Os carregadores de veículos elétricos podem ser \emph{on-board} ou \emph{off-board}. No primeiro caso, o veículo é conectado diretamente à rede elétrica e o circuito de potência interno do automóvel realiza a retificação da tensão de entrada CA para CC. No último caso, as estações de carregamento realizam o processo de retificação e conversão de energia e fornecem para o veículo uma tensão de alimentação CC.

Considerando os desafios existentes para a utilização em massa de veículos elétricos, este trabalho visa projetar e simular um carregador \emph{on-board} trifásico bidirecional completo, incluindo tanto o estágio do PFC quanto o estágio do conversor CC-CC.
